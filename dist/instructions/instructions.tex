\documentclass[10pt,a4paper]{article}
\usepackage[utf8]{inputenc}
\usepackage[french]{babel}
\usepackage[T1]{fontenc}
\usepackage{hyperref}

\begin{document}

\pagenumbering{gobble}

\begin{center}
{\Huge Instructions}
\end{center}

\textit{Note: } Le jeu ne fonctionne pas sur les navigateurs Safari et Internet Explorer. Il fonctionne cependant sur Google Chrome, Firefox et Opera.

\vspace{15pt}

Voici comment lancer le jeu de mon TM:

\vspace{10pt}

\textbf{Sur Microsoft Windows}
\begin{itemize}
\item Ouvrez le dossier "web\_src"
\item Double-cliquez sur "lanceur-win.exe"
\item Si Windows vous demande d'accepter le lancement du programme, faites-le
\item Normalement, une fenêtre de navigateur avec le jeu devrait se lancer
\end{itemize}

\vspace{15pt}

\textbf{Sur macOS}
\begin{itemize}
\item Ouvrez le dossier "web\_src"
\item Double-cliquez sur "lanceur-mac.app"
\item Normalement, une fenêtre de navigateur avec le jeu devrait se lancer
\end{itemize}

\vspace{15pt}

Le programme utilisé pour créer le serveur local nécessaire au fonctionnement du jeu s'appelle \textit{Mongoose Web Server} et est développé par l'entreprise \textit{Cesanta}. Il est disponible à l'adresse suivante: \url{https://cesanta.com/binary.html}.

\end{document}